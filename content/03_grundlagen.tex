\chapter{Methoden}\label{make}

\section{Beugung langsamer Elektronen}
Zur Untersuchung periodischer Strukturen eignen sich Beugungsmethoden mit Teilchen am besten. Für das LEED-Experiment werden niederenergetische Elektronen
an Oberflächen gebeugt. Für eine atomare Auflösung, in der Größenordnung von einigen \si{\angstrom}, werden Elektronen in diesem Wellenlängenbereich benötigt.
Bei diesen Energien $E =  \SI{50-200}{\eV}$ führt auch zu einer geringen freien Weglänge, wodurch die Methode Oberflächensensitiv bleibt.

Der Aufbau des LEED-Experiments ist in Abbildung \ref{pic:LEED} dargestellt.
Für die Durchführung wird eine Elektronenkanone senkrecht auf die Oberfläche der Probe gerichtet, sodass die Elektronen in Richtung der Probe beschleunigt werden. An der Probe kommt es zu elastischer Beugung der Elektronen, welche dann geradlinig in Richtung des Leuchtschirms fliegen. Kurz vor dem Schirm befindet sich ein
Gitter, welches mit Hilfe einer Abbremsspannung einen großen Teil der inelastisch gestreuten Elektronen aussondert. Daraufhin werden die übrigen Elektronen durch ein weiteres Gitter auf den Schirm beschleunigt, um
dort durch Lumineszenz die Beugungsreflexe sichtbar zu machen, welche durch eine Kamera aufgenommen werden.
\begin{figure}
    \centering
    \includegraphics[scale=1]{./Pics/LEEDaufbau.png}
    \caption{Schematischer Aufbau eines LEED-Experiments.\cite{Oberflächenphysik}}
    \label{pic:LEED}
\end{figure}
